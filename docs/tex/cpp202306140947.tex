%%%%%%%%%%%%%%%%%%%%%%%%%%%%%%%%%%%%%%%%%%%%%%%%%%%%%%%%%%%%%%%%%%%%%%%%%%%%%%%%%%%%%%%%%%%%%%%%%%%%%%%%%%
%Write by:ShuwenHe
%Date:20230613
%%%%%%%%%%%%%%%%%%%%%%%%%%%%%%%%%%%%%%%%%%%%%%%%%%%%%%%%%%%%%%%%%%%%%%%%%%%%%%%%%%%%%%%%%%%%%%%%%%%%%%%%%%

%%%%%%%%%%%%%%%%%%%%%%%%%%%%%%%%%%%%%%%%%%%%%%%%%%%%%%%%%%%%%%%%%%%%%%%%%%%%%%%%%%%%%%%%%%%%%%%%%%%%%%%%%%
\documentclass[12pt,twiside,a4paper]{ctexbook}
\usepackage[centertags]{amsmath}
\usepackage{amsfonts}
\usepackage{amsthm}
\usepackage{newlfont}
\usepackage{makeidx}
\usepackage{wasysym}
\usepackage{geometry} 
\usepackage{graphics}
\usepackage{slashbox} 
\usepackage{fancyhdr} 
\usepackage[pdftex]{graphicx}
\usepackage{epstopdf}
\usepackage{cite}

\usepackage[numbers,sort&compress]{natbib}

\setlength\parskip{\baselineskip}
\setcounter{tocdepth}{0}
\setcounter{secnumdepth}{4}
\renewcommand\thesection{\arabic{section}}
\usepackage[pdfstartview=FitH,CJKbookmarks=true,bookmarks,bookmarksnumbered=true,
    colorlinks=true,citecolor=black,linkcolor=black,anchorcolor=green,urlcolor=black]{hyperref}
\usepackage{titlesec}
\titleformat{\chapter}[display]{\normalfont\huge\bfseries\center}{\chaptertitlename}{1pt}{\Huge}
\titleformat{\section}{\normalfont\Large\bfseries}{\thesection}{1em}{}
\titleformat{\subsection}{\normalfont\large\bfseries}{\thesubsection}{1em}{}
\titleformat{\subsubsection}{\normalfont\normalsize\bfseries}{\thesubsubsection}{1em}{}
\titleformat{\paragraph}[runin]{\normalfont\normalsize\bfseries}{\theparagraph}{1em}{}
\titleformat{\subparagraph}[runin]{\normalfont\normalsize\bfseries}{\thesubparagraph}{1em}{}
\titlespacing*{\chapter} {0pt}{10pt}{10pt}
\titlespacing*{\section} {0pt}{0.5ex plus 1ex minus .2ex}{0.3ex plus .2ex}
\titlespacing*{\subsection} {0pt}{0.25ex plus 1ex minus .1ex}{0.5ex plus .1ex}
\titlespacing*{\subsubsection}{0pt}{3.25ex plus 1ex minus .2ex}{1.5ex plus .2ex}
\titlespacing*{\paragraph} {0pt}{3.25ex plus 1ex minus .2ex}{1em}
\titlespacing*{\subparagraph} {\parindent}{3.25ex plus 1ex minus .2ex}{1em}
\numberwithin{chapter}{part}
\geometry{left=2.0cm,right=20mm,top=25mm,bottom=25mm}
\let\cleardoublepage\clearpage
%%%%%%%%%%%%%%%%%%%%%%%%%%%%%%%%%%%%%%%%%%%%%%%%%%%%%%%%%%%%%%%%%%%%%%%%%%%%%%%%%%%%%%%%%%%%%%%%%%%%%%%%%%

%%%%%%%%%%%%%%%%%%%%%%%%%%%%%%%%%%%%%%%%%%%%%%%%%%%%%%%%%%%%%%%%%%%%%%%%%%%%%%%%%%%%%%%%%%%%%%%%%%%%%%%%%%
%mathematics
\usepackage{amssymb}
\usepackage{diagbox}
%%%%%%%%%%%%%%%%%%%%%%%%%%%%%%%%%%%%%%%%%%%%%%%%%%%%%%%%%%%%%%%%%%%%%%%%%%%%%%%%%%%%%%%%%%%%%%%%%%%%%%%%%%

%%%%%%%%%%%%%%%%%%%%%%%%%%%%%%%%%%%%%%%%%%%%%%%%%%%%%%%%%%%%%%%%%%%%%%%%%%%%%%%%%%%%%%%%%%%%%%%%%%%%%%%%%%
%
%%%%%%%%%%%%%%%%%%%%%%%%%%%%%%%%%%%%%%%%%%%%%%%%%%%%%%%%%%%%%%%%%%%%%%%%%%%%%%%%%%%%%%%%%%%%%%%%%%%%%%%%%%

%%%%%%%%%%%%%%%%%%%%%%%%%%%%%%%%%%%%%%%%%%%%%%%%%%%%%%%%%%%%%%%%%%%%%%%%%%%%%%%%%%%%%%%%%%%%%%%%%%%%%%%%%%
%
\usepackage{tipa}
%%%%%%%%%%%%%%%%%%%%%%%%%%%%%%%%%%%%%%%%%%%%%%%%%%%%%%%%%%%%%%%%%%%%%%%%%%%%%%%%%%%%%%%%%%%%%%%%%%%%%%%%%%

%%%%%%%%%%%%%%%%%%%%%%%%%%%%%%%%%%%%%%%%%%%%%%%%%%%%%%%%%%%%%%%%%%%%%%%%%%%%%%%%%%%%%%%%%%%%%%%%%%%%%%%%%%
\begin{document}
%%%%%%%%%%%%%%%%%%%%%%%%%%%%%%%%%%%%%%%%%%%%%%%%%%%%%%%%%%%%%%%%%%%%%%%%%%%%%%%%%%%%%%%%%%%%%%%%%%%%%%%%%%

\author
{
Peking University\\
北京大学\\
ShuwenHe\\
何书文
}

%%%%%%%%%%%%%%%%%%%%%%%%%%%%%%%%%%%%%%%%%%%%%%%%%%%%%%%%%%%%%%%%%%%%%%%%%%%%%%%%%%%%%%%%%%%%%%%%%%%%%%%%%%
\title{cpp}
\maketitle
\tableofcontents
\pagestyle{fancy}
%%%%%%%%%%%%%%%%%%%%%%%%%%%%%%%%%%%%%%%%%%%%%%%%%%%%%%%%%%%%%%%%%%%%%%%%%%%%%%%%%%%%%%%%%%%%%%%%%%%%%%%%%%

\lhead{\includegraphics{shuwenedu.png}}
\rhead{科技特长生升学规划}
\lfoot{\includegraphics{pku.png}算法第一人北大何书文}
\rfoot{改变您家孩子命运的老师}
%%%%%%%%%%%%%%%%%%%%%%%%%%%%%%%%%%%%%%%%%%%%%%%%%%%%%%%%%%%%%%%%%%%%%%%%%%%%%%%%%%%%%%%%%%%%%%%%%%%%%%%%%%

%%%%%%%%%%%%%%%%%%%%%%%%%%%%%%%%%%%%%%%%%%%%%%%%%%%%%%%%%%%%%%%%%%%%%%%%%%%%%%%%%%%%%%%%%%%%%%%%%%%%%%%%%%
\chapter{oi plan}
信息学竞赛家长会\\
何老师你能介绍下自己吗?\\
答:\\
自我介绍\\
北京大学硕士\\
研究方向:算法设计与分析涉及到很多数学知识比如离散数学,我在北京大学学习期间花时间最多的地方就是北京大学图书馆,亚洲最大的图书馆,每天早晨6点到北京大学图书馆3楼排队,抢座位,有的同学会问为什么座位需要抢呢?北京大学不是号称亚洲最大的图书馆吗?为啥还要抢座位呢?当时文具盒小学一年级到北京大学之前用的铁的生锈发黑的锈色斑斑,学英语的,英文字典翻烂了。\\
我给大家讲几个有发生在北京大学图书馆的故事\\
日本人排队\\

姚班\\
姚期智\\
隔壁清华大学 中央主楼2次报告\\
讲了一些与姚老师合作的科学家\\
其中包括数学家计算机科学家\\
高德纳

\chapter{全国青少年信息学奥林匹克系列竞赛大纲}
全国青少年信息学奥林匹克系列竞赛大纲(2023 年修订版)\\
CCF NOI 科学委员会全体审定\\
CCF是中国计算机学会China Computer Federation\\
NOI全国青少年信息学奥林匹克系列竞赛National Olympiad in Informatics
全国青少年信息学奥林匹克竞赛(NOI)于1984 年
创办,至今已经走过了近四十个年头。NOI 从无到有,
通过持续发展革新,已经从单一的小规模赛事发展为形
式丰富多样、参加人数众多的系列性活动,每年参加的
学生达十余万,选拔出的选手在国际赛场上屡创佳绩。
从NOI 活动走出的大批选手,在名校接受了很好的学术
训练,毕业后在计算机专业领域大显身手,很多已经成
长为技术领军人才,为社会做出了突出的贡献。\\
然而,随着计算机技术的发展,特别是人工智能技
术即将全面深入到我们的日常工作和生活,我们还需要
更多的计算机专业优秀人才,这就给NOI 活动提出了新
的挑战和要求。我们清醒地认识到,NOI 活动的普及面
还不够广,特别是合格师资的缺口还很大,不少学校无
法配备专业的指导教师,这就为活动发展带来了障碍。
此外,尽管NOI 活动已开展多年,但在知识考察的范围
方面还缺乏一个明确的规范,其知识边界主要靠组织者
和指导教师的既往经验来把握,这就难免会让活动的开
展和选手的成长走一些弯路。这当然是我们不愿意看到
的。\\
为确保NOI 活动的发展和选手的学习有章可循,并
尽快培养出更多的胜任师资,CCF NOI 科学委员会与全国
NOI 界数十位指导教师共同努力,于2021 年制定完成并
首次发布了NOI 大纲。大纲发布后,对竞赛组织、教师
教学和选手学习发挥了很好的引导作用,但是也发现了一
些不足,需要尽快完善。大纲工作组经过近一年的努力,
在广泛吸取各方意见的基础上,完成了本次修订。\\
我相信这部大纲对NOI 以及计算机科学普及等活动
将会持续发挥积极的引导作用。同时,NOI 科学委员会也
秉承CCF 一向坚持的开放思路,诚挚希望广大指导教师
和选手能够在学习和竞赛的过程中,将发现的问题及时反
馈给大纲工作组或NOI 科学委员会(noi@ccf.org.cn),
以便我们更好地完善这部大纲。\\
我非常感谢NOI 科学委员会和相关指导教师对大纲
的完善所付出的努力,特别感谢大纲工作组组长赵启阳博
士从大纲的起草到修订所做出的重要贡献。\\
全国青少年信息学奥林匹克系列竞赛大纲\\
\section{简介}
本大纲的制定目的在于:\\
(1) 为全国青少年信息学奥林匹克(National
Olympiad in Informatics,NOI)系列竞赛以及中国计算
机学会(China Computer Federation,CCF)主办的其他
有关活动的题目命制提供依据;\\
(2)为NOI 指导教师的教学提供方向和指导;\\
(3)为参加NOI 系列竞赛、CCF 主办的其他有关
活动的学生和信息学爱好者的学习提供范围;\\
(4)为各省市开展和组织NOI 省选等活动提供
参照。
\section{原则}
\subsection{等级化原则}
按照目前NOI 系列活动开展的现状,以及将来可
能的发展,大纲将各知识点分成入门级、提高级和NOI
级。高级别自动包含低级别知识点。各级别与NOI 以
及CCF 主办的其他有关活动的对应关系如下:\\
(1)入门级:CCF 非专业级软件能力认证入门组
(Certified Software Professional Junior,简称CSP-J);\\
(2) 提高级: 全国青少年信息学奥林匹克联
赛(National Olympiad in Informatics in Provinces,
NOIP)、CCF 非专业级软件能力认证提高组(Certified
Software Professional Senior,简称CSP-S);\\
(3)NOI 级:全国青少年信息学奥林匹克竞赛
(NOI)及以上,包括国际信息学奥林匹克(International
Olympiad in Informatics,IOI) 中国队选拔(CTS)、
NOI 冬令营、国家集训队集训等。\\
除上述等级以外,还对所有知识点标定了学习难度
系数(范围为1~10)。考虑到相邻级别中知识点的难
度系数范围可能互有交叉,入门级知识点难度系数范围
取1~5,(除入门级知识点外的)提高级知识点难度系
数范围取5~8,(除入门级、提高级知识点外的)NOI
级知识点难度系数范围取7~10。
各知识点难度系数以【X】的格式列在知识点之前。
\subsection{差异化原则}
为促进信息学和NOI 活动的普及,大纲应较详尽
地规定中低等级知识点的范围,以尽可能清晰地划定相
应等级的知识范围,有效地指导入门学生的学习及相关
的教学活动;为保证和促进我国选手在IOI 竞赛中的竞
争力,大纲应避免过于严格地限制命题的思路,须为
NOI 等高水平竞赛的题目命制留有充分的开放性,因此
不宜过于细致地规定高等级知识点的范围。为此,大纲
在制定中将采取“上粗下细”的指导思想:知识等级越低,
其内容规定得越细;知识等级越高,其内容规定得越粗。
\subsection{统一性原则}
为保证大纲的简明性和系统性,高等级比赛的知识
范围将自动地包含低等级比赛的所有知识点。同时,对
每个等级按照竞赛环境(Linux 和Windows)、程序设
计语言(C++)、数据结构、算法以及数学等进行了分类。
对每个大类又按照知识点的属性继续划分为若干小类;
某些知识点可能与多个类别均有紧密或松散联系,本大
纲均按其主要属性划定其类别,以避免同一知识点在多
个类别中的重复出现。
\subsection{建议}
建议在各级别竞赛题目的命制中,
(1)各级别竞赛或活动的考察范围不超过对应的
大纲级别,其中难度系数为10 的知识点仅用于CTS;
(2)避免对算法复杂度的常系数的考察;
(3)部分单个知识点可能对应不同层次、不同性
能的多个数据结构或算法。考察内容应以常见的、经典
的内容为主,避免虽具有微弱性能优势(例如算法复杂
度的细微改进)但较为冷僻或过新的数据结构和算法。
\section{大纲}
入门级\\
基础知识与编程环境\\
1.计算机的基本构成(CPU、内存、I/O 设
备等)\\
2.Windows、Linux 等操作系统的基本概念
及其常见操作\\
3.计算机网络和Internet 的基本概念\\
4.计算机的历史和常见用途\\
5.NOI 以及相关活动的历史\\
6.NOI 以及相关活动的规则\\
7.位、字节与字\\
8.程序设计语言以及程序编译和运行的基
本概念\\
9.使用图形界面新建、复制、删除、移动
文件或目录\\
10.使用Windows 系统下的集成开发环境
( 例如Dev C++ 等)\\
11.使用Linux 系统下的集成开发环境例
如Code::Blocks 等)
12.【 1 】g++、gcc 等常见编译器的基本使用

\section{C++程序设计1}
\subsection{程序基本概念}
1.标识符、关键字、常量、变量、字符串、
表达式的概念\\
2.常量与变量的命名、定义及作用\\
3.头文件与名字空间的概念\\
4.编辑、编译、解释、调试的概念

\subsection{基本数据类型}
1.整数型:int 、 long long\\
2.实数型:float、 double\\
3.字符型:char\\
4.布尔型:bool

\subsection{字符串的处理}
字符数组与相关函数\\
string 类与相关函数

\subsection{程序基本语句}
1.cin 语句、scanf 语句、cout 语句、printf
语句、赋值语句、复合语句\\
2.if 语句、switch 语句、多层条件语句\\
3.for 语句、while 语句、do while 语句\\
4.多层循环语句

\subsection{基本运算}
1.算术运算:加、减、乘、除、整除、求余\\
2.关系运算:大于、大于等于、小于、
小于等于、等于、不等于\\
3.逻辑运算:与、或、非\\
4.变量自增与自减运算\\
5.三目运算\\
6.位运算:与、或、非、
异或、左移、右移

\subsection{数学库常用函数}
绝对值函数、四舍五入函数、下取整函数、
上取整函数、平方根函数、常用三角
函数、对数函数、指数函数

\subsection{结构化程序设计}
顺序结构、分支结构和循环结构\\
自顶向下、逐步求精的模块化程序设计\\
流程图的概念及流程图描述

\subsection{数组}
数组与数组下标\\
数组的读入与输出\\
二维数组与多维数组

\subsection{函数与递归}
1.函数定义与调用、形参与实参\\
2.传值参数与传引用参数\\
3.常量与变量的作用范围\\
4.递归函数

\subsection{结构体与联合体}
1.结构体\\
2.联合体

\subsection{指针类型}
1.指针\\
2.基于指针的数组访问\\
3.字符指针\\
4.指向结构体的指针

\subsection{文件及基本读写}
1.文件的基本概念、文本文件的基本操作\\
2.文本文件类型与二进制文件类型\\
3.文件重定向、文件读写等操作

\subsection{STL 模板}
1.算法模板库中的函数:min、max、swap、sort\\
2.栈(stack)、队列(queue)、链表(list)、向量(vector)等容器

\section{数据结构}
\subsection{线性结构}
1.链表:单链表、双向链表、循环链表\\
2.栈\\
3.队列
\subsection{简单树}
1.树的定义与相关概念\\
2.树的表示与存储\\
3.二叉树的定义与基本性质\\
4.二叉树的表示与存储\\
5.二叉树的遍历:前序、中序、后序

\subsection{特殊树}
1.完全二叉树的定义与基本性质\\
2.完全二叉树的数组表示法\\
3.哈夫曼树的定义和构造、哈夫曼编码\\
4.二叉搜索树的定义和构造

\subsection{简单图}
1.图的定义与相关概念\\
2.图的表示与存储:邻接矩阵\\
3.图的表示与存储:邻接表


\subsection{}
\subsection{}
\subsection{}
\subsection{}


\chapter{让我改变命运的两位老师}
\section{宋发道}
\section{屈婉玲}
\href{https://bbs.pku.edu.cn/v2/post-read.php?bid=176&threadid=17762555}\\
大家好,我是何书文,昨天北大校友同学发了上面的链接给我,说是我导师,我打开看了,既难受又震惊!导师走了!去年导师组同学跟我说屈老师得了一场重病,动了手术,疫情之前我跟一个导师组同学谈事情时提过导师动手术的事,还相约要回去看看导师,没想到这么快导师就走了!    或许我们每个人都很忙,忙于自己的事情,身边的亲人老师慢慢离我们而去,留给我们的或许只有深深的遗憾与悲痛之情,让我们时常感觉时间过得太快,人生如此之短暂。希望看到此文章的同学,是屈老师带过的学生,可以在文章后面回复与屈老师相关的点点滴滴,让我们作为北大的学生们,来一起怀念这位伟大而朴素的老师!    屈老师两次被北大学生评为北京大学十佳教师,在北大是一位深受学生爱戴的老师。屈老师永远是我人生的榜样,其为人处事会永远深深的影响着我,这就是榜样的力量。我相信屈老师走了以后还有我们这些学生把她老人家的教育事业继续推进。    有时我在想我们眼下的社会是一个信息爆炸的社会,盛行着各种信息的媒体:如抖音、今日头条、微信、微博、facebook、twitter等。我在想这个社会有时是不是过于浮躁,大家都会去追求各自的成功,此时我们回想导师的点点滴滴,心情会慢慢放松了下来。让我们一起回忆怀念我们敬爱的老师!\\          
希望有一天我们能有能力像Elon Musk那样为人类的文明延续做些贡献,我们相信我们能!因为有一股平静而有力的力量在一直指引我们一路前行!我们应该时时刻刻鞭策自己要做出伟大的产品为宇宙做出力所能及的贡献!相对于宇宙我们每个人的个体渺小到可以忽略不计,但我们还是会奋力一搏,我们相信我们能!因为有您指引我们一路向前!!!\\
一、生平简介    屈婉玲,北京大学信息科学技术学院教授、博士生导师,中国人工智能学会离散数学专委会委员。主要研究方向是算法设计与分析,发表论文20余篇,出版教材、教学参考书、译著20余本,其中包含多部国家级规划教材和北京市竞品教材。所讲授的离散数学课程被评为国家精品课程,两次被评为北京大学十佳教师,并获得北京市优秀教师称号。曾主持过多项国家教材和课程建设项目,并获得北京市教育教学成果(高等教育)一等奖,系国家精品课“离散数学”课程主持人,“算法设计与分析”课程主持人。    屈老师获得2017CCF夏培肃奖(\href{https://www.ccf.org.cn/c/2018-02-05/622953.shtml})。夏培肃老师是中国计算机之母,师从华罗庚,她研究了107机,成就了传奇(2014年去世\href{http://www.cas.cn/xw/yxdt/201408/t20140827_4193177.shtml})。60年前,夏老师是毛泽东主席接见的科学家团队的成员之一。 2012年夏老师还健在时,我给夏老师发了一封关于量子算法研究的邮件,夏老师把该邮件转给了她的学生胡伟武(龙芯中科技术有限公司董事长、中国科学院计算技术研究所研究员\href{https://iat.ustc.edu.cn/Practical-tutor/2318.html},胡老师还专门给我回复并进一步沟通了相关研究。)前几辈学者的学术与研究作风都值得我们深入学习的,我现在想想很多事情都是相关联的。\\
二、开学吃饭与答谢晚宴    我们研究生开学时,导师组同学说请屈老师吃饭,屈老师算法课下课后说我们去食堂吃吧。记得当时我们几个导师组的同学来到食堂要给屈老师打餐,屈老师说我自己有带的饭卡,屈老师总是为我们学生着想,怎么可能让我们学生出钱请客呢,这些都是小事,但能反应一个老师的品质。    在食堂吃饭时屈老师回忆起当年自己的在北大物理系学习的经历,毕业后留校工作文革时期,被下放分配工作,开过叉车,对此,现在的我们简直难以想象。文革后屈老师回归北大教授高等数学,是北大第一批研究离散数学的老师,当时国内没有教材,其只能自己编写教材。屈老师讲当时的北大老师都有奉献精神,有一个老师为了研究数学,终身未婚(跟牛顿类似)一生献给了科学。后来北大建立计算机系,没有算法教材,屈老师又开始编写算法设计与分析(https://item.jd.com/12570866.html)教材给学生上课用。屈老师说:“回忆在北大学生时代学习物理,后来研究数学与算法,其实都是相通的,没什么太大区别。”现在想想也是,最关键是找到研究的方法,方法对了,研究问题的思维思路对了,剩下的问题就不是问题了。现在想想,决定一个人的高度最重要的还是思维。\\
    我记得研究生阶段与导师唯一一次算是比较正式的吃饭是到我们论文答辩结束毕业时,导师组同学组织的一次在北大西南门的答谢晚宴。\\
三、上课衣着    屈老师上课时衣着朴素无华,在算法设计与分析(\href{https://www.icourse163.org/course/PKU-1002525003})课堂上几个小时一直站着上课。我作为一名年轻教师连续站着3个小时一直讲课板书都会觉得累,更何况是对于年近古稀的屈老师,确实非常值得钦佩。
四、离散数学    屈老师离散数学(\href{https://item.jd.com/11633329.html})\\
五、算法设计与分析    屈老师的算法设计与分析课,我是课代表,要给上课的学生订购教材。因为屈老师的算法课程是必修课程,所以学生人数较多,一般都在大教室上课,二教一楼的大教室,能坐多少学生想必大家应该都知道。屈老师的课,大教室都座无缺席,我清晰记得当时购买算法设计与分析教材很紧俏,我在京东、当当、亚马逊平台都购买了,可是教材数量还是不够,后来屈老师了解情况后联系了出版社。    算法课在线课程答疑,屈老师做的很好,尽心尽责,其实答疑是教学之外的事情,屈老师觉得答疑对于我们学生很有意义,可以与学生互动,解决学生近期算法教材上学习遇到的问题,从这点来讲,是非常可贵的,我们可以想想是抽出老师在家休息的时间来给学生免费答疑,是多么的可贵。一次答疑要好几个小时,屈老师总是忘我的工作着,她是那么的平易近人,只为她的学生着想却牺牲了自己休息的时间。
六、数学暑期学校    北京大学开办了应用数学暑期学校,当时我对数学系组织的暑期数学研讨班感兴趣,要求是要有导师签字,并且有相关专家推荐。为了参加项目,头天晚上23:47我给屈老师发了邮件,说明我想参加这个数学项目,第二天上午09:01屈老师给我回了邮件说”今天中午1:00到我办公室(1625/理科一号楼)谈谈”。由于那段时间我整天在图书馆看相关的数学文献,所以当天没有带电脑,也没有及时收邮件。下午3:53估计是看我没回邮件,也没来办公室,又给我回了邮件:“今天中午我一直在办公室,但你没有来。如果你还找我的话,明天上午10:00以后,我还在办公室等你”。直到晚上10:50我从图书馆回去打开电脑看到屈老师给我回复了两封邮件,我当时心里很是惭愧,让屈老师她老人家在办公室等了我一个下午,看完老师的邮件我马上回了邮件:“不好意思,我现在才看到邮件,今天在图书馆没带电脑,那我明天上午10:00以后去您办公室”。为了表示我的歉意,我第二天给导师带了些苹果还有其他水果,屈老师怎么也不收下,说不需要。后来导师觉得盛情难却,只收下了两个火龙果,代表领了我的心意。现在想想,屈老师确实不缺这点水果,但那确实是我作为学生的心意,表达对导师的敬意,同时表示我对没有及时查邮件,让导师白白等了半天表示真诚的歉意。
七、毕业论文    屈老师给我们导师组几个学生批改论文,每一句都会细心标记,一一罗列出来,真的可以细致到标点符号,对屈老师的付出真的很感动。\\
八、创业辅导    当初我创立北京书文创新教育科技有限公司时,心里也是想成为导师那样受人尊敬的老师,所以我选择了将教育作为创业切入点,由于喜欢数学,所以从数学切入,当然也有其他学科。    开始创业时我犯了一些错误,导师及时发了邮件纠正了我的错误。刚创业的头一两年我们做了一个教育视频网站,也可以发相关的课件,我们的初衷是使得教育更公平。在中国西部一些偏远地区,教育问题还是不能很好地解决,当时移动app刚刚兴起智能手机还未普及,所以我们当时开发的是web,但只要有网络就可以访问。目前中国确实有很多学校的师资力量和教育资源相对比较丰富,但仍然有一些地方院校的师资力量和教育资源比较欠缺。这个是事实。着眼于这一点,我们当时想做的事情,就是为解决教育不公平做一些力所能及的事情。于是乎我们当时想我们上课享受北大这么好的资源,这么好的老师,那我们能否把这些资源共享给一些偏远地区的想学习的学生呢,于是乎我们上传了一些老师的教学讲义。我们的出发点是好的,为了把这个事情做好,当时我们做了一些宣传。    导师知道这些事情给我们写了封邮件,大概意思:“北大老师的课件上网发布,需要征得老师的同意,需要签署正式的合同。如果你没有征求过老师的意见,自己就这样做了,这是违反国家关于保护知识产权规定的。自主创业是一件好事,国家也支持自主创业,但是应该遵守国家的法律”。现在想想,确实感谢导师的提醒,即使是校友,即使出发点是好的,有时好心会办成坏事,于是我们下架了北大相关的资源。在这件事情中我从导师身上学到了做人做事一定要有自己的标准,我们北大人做事要有一定的原则,有些事可以做,有些事一定要想清楚了再做。    后来我们想做一些“翻转课堂”的事情,跟导师也探讨过翻转课堂的好处,那时我们在互联网教育创新中心租了一大间办公场地,那里有相关高端的录制设备,我们当时跟优酷也有合作,当然现在B站上面的学习资料及其腾讯课堂上的视频做得都比优酷好,优酷自从被阿里收购后,我们从用户体验的角度感觉到优酷没有上面提到的平台发展的快。互联网公司的发展确实日新月异,现在也出现了一些独角兽互联网公司。当时我想跟导师探讨是否可以与北大进行合作,导师给了我们一些建议,大概意思如下:“翻转课堂是一种新的教育模式,我也听说过。北大一直在推动MOOC教学的进展,在高校中也算做得比较早的,有关成果已经评选为教育部的十大进展之一。目前北大有两个平台: Coursera和Edx,是与国外名校合作的平台,目前有几十门课程上线了。信息学院和软件学院都有课程在上面。今年春天,也就是这个学期,我的算法基础课程在Coursera上线,大约10周,70多个视频,配合每周的测试。目前刚刚结束,本周内完成考试。通过考试的学生会拿到北大的证书。我们的MOOC课程已经纳入北京大学教学改革的整体规划,今后学校还支持更多的课程在这两个平台上上线。我想,老师的教学与学校的支持分不开,与你们公司合作涉及的恐怕不是老师个人的事,应该是学校与公司之间的合作。如果你有想法,可以找学校谈谈”。从这些方面来讲,创业的路上屈老师也给到了我们很多思路。    我不仅仅感谢屈老师上面述及的对我们学生的帮助。即使我们毕业了,离开了学校,我作为一个学者依然可以与屈老师探讨学术问题,确切的讲是向屈老师请教问题,是一件幸运的事请。只要是与屈老师研究方向相关的问题,屈老师一定会及时给予点拨与解答,如果不是屈老师研究的方向,她也会介绍北大别的相关领域的知名教授帮助学生解答相关问题。上面我提到过,还记得刚读研究生的时候,我给中科院夏培肃老师发过邮件,还有因为我在清华听过姚期智老师的讲座所以也给姚教授写过关于量子学术方面的邮件。当时我对量子力学、量子计算机相关学术问题特别感兴趣,想从事理论计算机科学方面的研究工作,于是我与屈老师探讨过这方面的学术问题,她给我回复的大概意思是说:“我对量子算法不了解,校本部信息科学技术学院的刘田副教授做过一些研究。他和我在一个办公室,我找找他,看看能不能约个时间,你们谈谈”。这个就是导师的可贵之处,即使是自己不擅长的领域,只要自己的学生感兴趣愿意研究相关领域,屈老师也会把身边相关的资源介绍给自己的学生。屈老师想让自己的学生能在学术上有所发展,只要我们愿意研究,能帮的一定会帮。这一点确实令我们感动,我相信屈老师的学生应该都有所体会。    我创业之后,作为一名老师给准备考清华的学生上课时,讲到棋盘多项式相关课题,这必定是离散数学的相关课题,是屈老师所擅长的。学生在学习时想得到各种证明的方法,并找到最优解。我发邮件与屈老师探讨这个话题,即使是毕业之后,即使正值暑假,屈老师也给了及时且耐心的回复。我是7月26日22:37(看上面给导师写的邮件可以看出来,创业确实需要热情,我今天为了写这篇文章特意查看了邮件收发时间,创业前期基本我都是凌晨才休息,我觉得创业前期就是要有这股冲劲,这样才有机会做成一些事情)给导师发的邮件,屈老师8月1号上午9:05给我回复的邮件,关于数学学术方面的讨论短短的几天时间就给回复,这个已经算是很快的了,更何况是暑假。导师有可能做些自己的学术研究几天都不开电脑查看邮件,或者也有可能暑期休息一下,这个都完全有可能。但屈老师还是给我回复得很及时,最让我感动的是屈老师给了我详细的数学证明,如下图。当然解决一个数学问题或者是算法问题的算法有好多种,但我们要从众多算法中找出最优解还是需要有一定的数学高度的。下图是导师给到的证明,我给清华的学生在课堂上进行了讲解,学生与家长都很满意。当然,未名BBS支持tex公式,做教育做数学的我写的好多数学教材用的都是tex,这里就直接截导师发我的doc证明过程了。    今天上午去了八宝山,看了屈老师最后一面,老师一路走好!
\section{孙悟空的师傅}
改变孙悟空命运的人\\
孙悟空的师傅是菩提祖师。在《西游记》中,孙悟空一开始是被关在山洞中的石猴,后来得到菩提祖师的指点,学会了七十二变、筋斗云等神通,成为了强大的斗士。菩提祖师也教导孙悟空为人处世之道,让他懂得正义和善良的本质。在后来的故事中,孙悟空经常念叨着菩提祖师的教诲,帮助他在取经路上磨练自己的心性,修行成为了一个更加完善的人和佛教修行者。\\
唐僧
\section{拉开孩子差距的是什么?}
孩子都是聪明孩子\\
信息差\\
规划\\
父母的引导

\section{陶哲轩}
前期母亲引导\\
后期自驱力

\section{佩雷尔曼}
前期母亲引导\\
后期自驱力

\section{韦东奕}
前期母亲引导\\
后期自驱力

\section{牛顿}
问:没爸没妈,牛爷为什么这么牛\\
答:自驱力

\section{C语言之父}

\section{Unix之父}
肯toms

\section{Linux之父}

\section{}
问:妈妈不行,爸爸不行的谁来引导?\\
答:答案是找到一位改变你孩子命运的老师\\
今天何老师就是改变你孩子命运的老师\\
先买电脑服务器级别7*24小时,放心电脑何老师一分钱不会挣的,京东多少钱,就给到大家多少钱,有的家长会说那我直接从京东买不就得了,我要给我们孩子装
系统,装开发环境,装各种软件呀,家长朋友们,而且我的这些付出何老师再次保证全部免费。\\

\section{欧拉}

\chapter{Artificial Intelligence人工智能}
\section{ChatGPT}
问:人工智能有必要学习吗?\\
答:人工智能必须要学习。

\section{工具}
问:何老师,普通人学人工智能用什么工具呢?\\
答:报了何老师人工智能课程的,免费宋人工智能工具。

\section{效率}
问:何老师,人工智能工具对我家孩子学习有什么帮助?\\
答:学了何老师的人工智能课程可以改变你家孩子的命运,因为何老师就是为了改变你家孩子命运而来的,用了何老师的人工智能工具学习,可以提高至少10倍的速度的效率学习。
我每天都用人工智能工具进行研究与学习,大大提高了我的研究与学习的效率,比如给孩子的讲义很多内容,人工智能都可以帮助到,用了何老师研发的人工智能工具相当于把
何老师一对一的请回家,手把手教你学习。你们知道吗,何老师一对一的数学课程计算机算法课程收费标准是多少吗?8年前是1000元/时,现在3000元/时,我现在都没时间。

\section{链表}
\subsection{单链表}
\subsection{双链表}
\subsection{循环链表}

\section{栈}
\section{双端栈}
\section{队列}
\section{双端队列}
\section{单调队列}
\section{优先队列}
\section{ST表 Sparse Table}

\chapter{树}
\section{二叉树的遍历}
\subsection{前序}
\subsection{中序}
\subsection{后续}

\section{完全二叉树}
\section{哈夫曼树}
\subsection{哈夫曼编码}
\subsection{二叉搜索树}

\section{平衡树}
\subsection{AVL}
\subsection{treap}
\subsection{splay}

\chapter{图}
\section{稀疏图}
\section{偶图}
\section{欧拉图}
\section{有向无环图}
\section{连通图与强连通图}
\section{双连通图}

\chapter{哈希表}
\section{数值哈希函数构造}
\section{字符串哈希函数构造}
\section{哈希冲突}

\chapter{分治算法}

\chapter{贪心法}

\chapter{递推法}

\chapter{递归法}

\chapter{二分法}

\chapter{倍增法}

\chapter{数值处理算法}

\chapter{排序}
\section{冒泡排序}
\section{选择排序}
\section{插入排序}
\section{计数排序}
\section{归并排序}
\section{快速排序}
\section{堆排序}
\section{桶排序}
\section{基数排序}

\chapter{字符串匹配}
\section{KMP算法}

\chapter{搜索}
\section{深度优先搜索}
\section{广度优先搜索}
\section{双向广度优先搜索}

\chapter{图论}
\section{深度优先遍历}
\section{广度优先遍历}
\section{泛红算法 flood fill}
\section{最小生成树}
\subsection{Prim}
\subsection{Kruskal}
\section{次小生成树}
\section{单源最短路}
\subsection{Bellman-Ford}
\subsection{Dijkstra}
\subsection{SPFA}
\section{单源次短路}
\section{Floyd-Warshall算法}
\section{有向无环图的拓扑排序}
\section{欧拉道路和欧拉回路}
\section{二分图}
\section{强连通分量}

\chapter{动态规划}
\section{一维动态规划}
\section{背包问题}
\section{区间类型动态规划}

\section{树型动态规划}

\section{状态压缩动态规划}

\clearpage
\end{document}
