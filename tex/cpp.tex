%%%%%%%%%%%%%%%%%%%%%%%%%%%%%%%%%%%%%%%%%%%%%%%%%%%%%%%%%%%%%%%%%%%%%%%%%%%%%%%%%%%%%%%%%%%%%%%%%%%%%%%%%%
%Write by:ShuwenHe
%Date:20230613
%%%%%%%%%%%%%%%%%%%%%%%%%%%%%%%%%%%%%%%%%%%%%%%%%%%%%%%%%%%%%%%%%%%%%%%%%%%%%%%%%%%%%%%%%%%%%%%%%%%%%%%%%%

%%%%%%%%%%%%%%%%%%%%%%%%%%%%%%%%%%%%%%%%%%%%%%%%%%%%%%%%%%%%%%%%%%%%%%%%%%%%%%%%%%%%%%%%%%%%%%%%%%%%%%%%%%
\documentclass[12pt,twiside,a4paper]{ctexbook}
\usepackage[centertags]{amsmath}
\usepackage{amsfonts}
\usepackage{amsthm}
\usepackage{newlfont}
\usepackage{makeidx}
\usepackage{wasysym}
\usepackage{geometry} 
\usepackage{graphics}
\usepackage{slashbox} 
\usepackage{fancyhdr} 
\usepackage{ulem} %下划线、删除线、波浪线
\usepackage[pdftex]{graphicx}
\usepackage{epstopdf}
\usepackage{cite}
\usepackage{listings}
\usepackage{tocbibind}
\usepackage[numbers,sort&compress]{natbib}

\setlength\parskip{\baselineskip}
\setcounter{tocdepth}{8} % 生成目录层级
\setcounter{secnumdepth}{4}
\renewcommand\thesection{\arabic{section}}
\usepackage[pdfstartview=FitH,CJKbookmarks=true,bookmarks,bookmarksnumbered=true,
    colorlinks=true,citecolor=black,linkcolor=black,anchorcolor=green,urlcolor=black]{hyperref}
\usepackage{titlesec}
\titleformat{\chapter}[display]{\normalfont\huge\bfseries\center}{\chaptertitlename}{1pt}{\Huge}
\titleformat{\section}{\normalfont\Large\bfseries}{\thesection}{1em}{}
\titleformat{\subsection}{\normalfont\large\bfseries}{\thesubsection}{1em}{}
\titleformat{\subsubsection}{\normalfont\normalsize\bfseries}{\thesubsubsection}{1em}{}
\titleformat{\paragraph}[runin]{\normalfont\normalsize\bfseries}{\theparagraph}{1em}{}
\titleformat{\subparagraph}[runin]{\normalfont\normalsize\bfseries}{\thesubparagraph}{1em}{}
\titlespacing*{\chapter} {0pt}{10pt}{10pt}
\titlespacing*{\section} {0pt}{0.5ex plus 1ex minus .2ex}{0.3ex plus .2ex}
\titlespacing*{\subsection} {0pt}{0.25ex plus 1ex minus .1ex}{0.5ex plus .1ex}
\titlespacing*{\subsubsection}{0pt}{3.25ex plus 1ex minus .2ex}{1.5ex plus .2ex}
\titlespacing*{\paragraph} {0pt}{3.25ex plus 1ex minus .2ex}{1em}
\titlespacing*{\subparagraph} {\parindent}{3.25ex plus 1ex minus .2ex}{1em}
\numberwithin{chapter}{part}
\geometry{left=2.0cm,right=20mm,top=25mm,bottom=25mm}
\let\cleardoublepage\clearpage
%%%%%%%%%%%%%%%%%%%%%%%%%%%%%%%%%%%%%%%%%%%%%%%%%%%%%%%%%%%%%%%%%%%%%%%%%%%%%%%%%%%%%%%%%%%%%%%%%%%%%%%%%%

%%%%%%%%%%%%%%%%%%%%%%%%%%%%%%%%%%%%%%%%%%%%%%%%%%%%%%%%%%%%%%%%%%%%%%%%%%%%%%%%%%%%%%%%%%%%%%%%%%%%%%%%%%
%mathematics
\usepackage{amssymb}
\usepackage{diagbox}
%%%%%%%%%%%%%%%%%%%%%%%%%%%%%%%%%%%%%%%%%%%%%%%%%%%%%%%%%%%%%%%%%%%%%%%%%%%%%%%%%%%%%%%%%%%%%%%%%%%%%%%%%%

%%%%%%%%%%%%%%%%%%%%%%%%%%%%%%%%%%%%%%%%%%%%%%%%%%%%%%%%%%%%%%%%%%%%%%%%%%%%%%%%%%%%%%%%%%%%%%%%%%%%%%%%%%
%
%%%%%%%%%%%%%%%%%%%%%%%%%%%%%%%%%%%%%%%%%%%%%%%%%%%%%%%%%%%%%%%%%%%%%%%%%%%%%%%%%%%%%%%%%%%%%%%%%%%%%%%%%%

%%%%%%%%%%%%%%%%%%%%%%%%%%%%%%%%%%%%%%%%%%%%%%%%%%%%%%%%%%%%%%%%%%%%%%%%%%%%%%%%%%%%%%%%%%%%%%%%%%%%%%%%%%
%
\usepackage{tipa}
%%%%%%%%%%%%%%%%%%%%%%%%%%%%%%%%%%%%%%%%%%%%%%%%%%%%%%%%%%%%%%%%%%%%%%%%%%%%%%%%%%%%%%%%%%%%%%%%%%%%%%%%%%

%%%%%%%%%%%%%%%%%%%%%%%%%%%%%%%%%%%%%%%%%%%%%%%%%%%%%%%%%%%%%%%%%%%%%%%%%%%%%%%%%%%%%%%%%%%%%%%%%%%%%%%%%%
\begin{document}
%%%%%%%%%%%%%%%%%%%%%%%%%%%%%%%%%%%%%%%%%%%%%%%%%%%%%%%%%%%%%%%%%%%%%%%%%%%%%%%%%%%%%%%%%%%%%%%%%%%%%%%%%%

\author
{
%Peking University\\
%北京大学\\
%ShuwenHe\\
%何书文\\
%1201220707@pku.edu.cn
Richard He \& Ritchie He
}

%%%%%%%%%%%%%%%%%%%%%%%%%%%%%%%%%%%%%%%%%%%%%%%%%%%%%%%%%%%%%%%%%%%%%%%%%%%%%%%%%%%%%%%%%%%%%%%%%%%%%%%%%%
%\centerline{\includegraphics{shuwenhe.png}}
%写好一本书:工匠精神!用心打造!夜深写于北京大学图书馆。作者亲自一线带课,所带学生多人保送或考入清华北大,根据多年清华附中、101中学、人大附中、北大附中、十一学校,考试真题分析经验所得。用此书考上心目中名校学生无数!何书文北京大学硕士,资深数学名师、信息学竞赛算法名师,所带学生多名考入人大附中早培、清华附中优才、101 实验班、北大附中实验班等名校。全国中学数学联赛、全国中学数学竞赛的辅导老师,全国NOI、CSP信息学竞赛辅导名师。何书文老师在北京大学学习期间立志从事教育事业,帮学生授业解惑。何书文老师小学期间学习奥数,并多次获奖,为以后的学习与研究打下良好基础。何书文 老师在中学阶段数学、物理均获奖。何书文老师在小学中学期间一直为数学课代表,中小学大学期间担任班长,何书文老师在北京大学被选为科技一苑苑长,组织北大同学积极参与校各项活动,积极参与校学生会工作,何书文老师被北京大学评为优秀入党积极分子.何书文老师经常参加北京大学数学课题的研讨班。何书文 老师是北京大学数学系暑期学校全国选出40 名优秀中青年数学人才之一,参加伦敦国王学院、美国杜克大学、美国纽约大学、加拿大多伦多大学教授组成的学术研讨班,研究PDE(偏微分方程),量子力学方面的数学课题的研究工作,并获得优异成绩结业。何书文老师作为项目经理用数学建模方法给大型企业开发软件,用数学方法规划提高企业产能协作效率。何书文 老师致力于数学方面的教学与研究工作,所带多名孩子已经被点优才进入清华附中创新班,101 实验班,人大附中早培班,是家长值得信赖的老师。考上学生继续跟随何书文老师学习全国数学联赛,全国数学竞赛系列课程,同时学习NOI、IOI、ACM算法编程竞赛。
%%%%%%%%%%%%%%%%%%%%%%%%%%%%%%%%%%%%%%%%%%%%%%%%%%%%%%%%%%%%%%%%%%%%%%%%%%%%%%%%%%%%%%%%%%%%%%%%%%%%%%%%%%

%%%%%%%%%%%%%%%%%%%%%%%%%%%%%%%%%%%%%%%%%%%%%%%%%%%%%%%%%%%%%%%%%%%%%%%%%%%%%%%%%%%%%%%%%%%%%%%%%%%%%%%%%%
\title{C++基础}
\maketitle
\tableofcontents % 显示目录
\newpage
\pagestyle{fancy}
%%%%%%%%%%%%%%%%%%%%%%%%%%%%%%%%%%%%%%%%%%%%%%%%%%%%%%%%%%%%%%%%%%%%%%%%%%%%%%%%%%%%%%%%%%%%%%%%%%%%%%%%%%

%\lhead{\includegraphics{shuwenedu.png}}
%\rhead{科技特长生升学规划 何校长 电话微信15010729356}
%\lfoot{\includegraphics{pku.png}算法第一人北大何书文}
%\rfoot{改变您家孩子命运的老师}
%%%%%%%%%%%%%%%%%%%%%%%%%%%%%%%%%%%%%%%%%%%%%%%%%%%%%%%%%%%%%%%%%%%%%%%%%%%%%%%%%%%%%%%%%%%%%%%%%%%%%%%%%%

%%%%%%%%%%%%%%%%%%%%%%%%%%%%%%%%%%%%%%%%%%%%%%%%%%%%%%%%%%%%%%%%%%%%%%%%%%%%%%%%%%%%%%%%%%%%%%%%%%%%%%%%%%
\chapter{include头文件}
\section{iostream}
\begin{lstlisting}[language=C++]
#include <iostream>

using namespace std;

int main(){
	cout<<"hello,world!"<<endl;
}
\end{lstlisting}

\section{fstream}
\begin{lstlisting}[language=C++]
#include <iostream>
#include <fstream>

using namespace std;

int main(){
	ifstream fin("number.in");
	ofstream fout("number.out");
	int n;
	fin>>n;
	fin.close();
	fout.close();
	return 0;
}
\end{lstlisting}

\chapter{variable变量}
\section{局部变量}
局部变量:在函数内部声明的变量。它们只能被函数内部语句使用。
\begin{lstlisting}[language=C++]
#include <iostream>

using namespace std;

int main(){
	int a,b,sum; // 局部变量声明
	a = 1,b = 2; // 实际初始化
	sum = a + b;
	cout<<"sum = "<<sum<<endl;
	return 0;
}
\end{lstlisting}

\section{全局变量}
局部变量和全局变量的名称可以相同,但是在函数内,局部变量的值会覆盖全局变量的值。
\begin{lstlisting}[language=C++]
#include <iostream>

using namespace std;

int i = 3;

int global_variable(){
	cout<<"global_i = "<<i<<endl;
	return 0;
}

int main(){
	global_variable();
	int i = 5;
	cout<<"main_i = "<<i<<endl;
	return 0;
}
\end{lstlisting}

\chapter{array数组}
\begin{lstlisting}[language=C++]
#include <iostream>

using namespace std;

int main(){
	int nums[3] = {1,2,3};
	for(int i = 0; i < sizeof(nums)/sizeof(nums[0]);i++){
		cout<<nums[i]<<endl;
	}
	return 0;
}
\end{lstlisting}

\chapter{vector向量}


\chapter{string字符串}
\section{CSP-J-2019-01number数字游戏}
【题目描述】\\
小$K$同学向小$P$同学发送了一个长度为8的01字符串来玩数字游戏,小$P$同学想要知道字符串中究竟有多少个1。\\
注意:01字符串为每一个字符是0或者1的字符串,如“101”(不含双引号)为一个长度为3的01字符串。\\
【输入格式】\\
输入文件名为number.in\\
输入文件只有一行,一个长度为8的01字符串$s$。\\
【输出格式】\\
输出文件名为number.out\\
输出文件只有一行,包含一个整数,即01字符串中字符1的个数。\\
【输入输出样例1】\\
\begin{tabular}{|c|c|}
\hline
number.in & number.out\\
\hline
00010100 & 2\\
\hline
\end{tabular}\\
【输入输出样例2】\\
\begin{tabular}{|c|c|}
\hline
number.in & number.out\\
\hline
11111111 & 8\\
\hline
\end{tabular}\\
解:\\
\includegraphics[width=0.5\textwidth]{number.png}
\begin{lstlisting}[language=C++]
#include <iostream>
#include <fstream>
#include <string>

using namespace std;

int main(){
	ifstream fin("number.in");
	ofstream fout("number.out");

	string s;
	fin>>s;
	int count = 0;

	for(int i = 0; i < s.length();i++){
		if(s[i] == '1'){
			count++;
		}
	}
	fout<<count;

	fin.close();
	fout.close();
	return 0;
}
\end{lstlisting}

\chapter{流程控制语句}
\section{for}
\subsection{template模板}
计算$1+2+3+\cdots+100 = ?$
\begin{lstlisting}[language=C++]
#include <iostream>

using namespace std;

int main(){
	int sum = 0;
	for(int i = 1;i <= 100;i++){
		sum += i;
	}
	cout<<"sum = "<<sum<<endl;
	return 0;
}
\end{lstlisting}

\section{if-else}
\begin{lstlisting}[language=C++]
// isPrime 判断是否素数
bool isPrime(int num) {
    if (num < 2)
        return false;
    for (int i = 2; i * i <= num; ++i) {
        if (num % i == 0)
            return false;
    }
    return true;
}

int prime(){
	bool b = isPrime(7);
	if (b == 1){
		cout<<"b是素数 "<<endl;
	}else{
		cout<<"b不是素数 "<<endl;
	}
	return 0;
}
\end{lstlisting}

\chapter{bit位运算}
\section{bit位运算}
$1\ll i$等价于 $2^i$。
\begin{lstlisting}[language=C++]
for(int i = 30;i >= 0; i--){
		int power = 1 << i;
}
\end{lstlisting}

\section{if else}
\subsection{CSP-J-2021-01-candy分糖果}
【题目背景】\\
红太阳幼儿园的小朋友们开始分糖果啦!\\
【题目描述】\\
红太阳幼儿园有$n$个小朋友,你是其中之一。保证$n\geq2$。\\
有一天你在幼儿园的后花园里发现无穷多颗糖果,你打算拿一些糖果回去分给幼儿
园的小朋友们。\\
由于你只是个平平无奇的幼儿园小朋友,所以你的体力有限,至多只能拿$R$块糖
回去。\\
但是拿的太少不够分的,所以你至少要拿$L$块糖回去。保证$n\leq L\leq R$。\\
也就是说,如果你拿了$k$块糖,那么你需要保证$L\leq k\leq R$。\\
如果你拿了$k$块糖,你将把这$k$块糖放到篮子里,并要求大家按照如下方案分糖
果:只要篮子里有\dotuline{不少于}$n$块糖果,幼儿园的所有$n$个小朋友(包括你自己)都从篮子
中拿走\dotuline{恰好}一块糖,直到篮子里的糖数量\dotuline{少于}$n$块。此时篮子里剩余的糖果均归你所有
——这些糖果是\dotuline{作为你搬糖果的奖励}。\\
作为幼儿园高质量小朋友,你希望让\dotuline{作为你搬糖果的奖励}的糖果数量(\dotuline{而不是你最后获得的总糖果数量}!)尽可能多;因此你需要写一个程序,依次输入$n, L, R$,并输出你最多能获得多少作为你搬糖果的奖励的糖果数量。\\
【输入格式】\\
从文件 candy.in 中读入数据。\\
输入一行,包含三个正整数 $n, L, R$,分别表示小朋友的个数、糖果数量的下界和上界。\\
【输出格式】\\
输出到文件 candy.out 中。\\
输出一行一个整数,表示你最多能获得的\dotuline{作为你搬糖果的奖励}的糖果数量。\\
【样例 1 输入】\\
7 16 23\\
【样例 1 输出】\\
6\\
【样例 1 解释】\\
拿$k = 20$ 块糖放入篮子里。\\
篮子里现在糖果数 $20 \geq n = 7$,因此所有小朋友获得一块糖;\\
篮子里现在糖果数变成 $13 \geq n = 7$,因此所有小朋友获得一块糖;\\
篮子里现在糖果数变成 $6 < n = 7$,因此这 6 块糖是\dotuline{作为你搬糖果的奖励}。
容易发现,你获得的\dotuline{作为你搬糖果的奖励}的糖果数量不可能超过 6 块(不然,篮子
里的糖果数量最后仍然不少于 n,需要继续每个小朋友拿一块),因此答案是 6。\\
【样例 2 输入】\\
10 14 18\\
【样例 2 输出】\\
8\\
【样例 2 解释】\\
容易发现,当你拿的糖数量 k 满足 $14 = L \leq k \leq R = 18$ 时,所有小朋友获得一块
糖后,剩下的 k − 10 块糖总是\dotuline{作为你搬糖果的奖励}的糖果数量,因此拿 $k = 18$ 块是最
优解,答案是 8。\\
【数据范围】\\
\begin{tabular}{|c|c|c|c|}
  \hline
  测试点 &  $n\leq$ & $R\leq$ & $R-L\leq$\\
  \hline
  1 & 2 & 5 & 5\\
  \hline
  2 & 5 & 10 & 10\\
  \hline
  3 & $10^3$ & $10^3$ & $10^3$\\
  \hline
  4 & $10^5$ & $10^5$ & $10^5$\\
  \hline
  5 & & & 0\\
  \cline{1-1}
  \cline{4-1}
  6 & $10^3$ & & $10^3$\\
  \cline{1-1}
  \cline{2-1}
  \cline{4-1}
  7 & $10^5$ & $10^9$ & $10^5$\\
  \cline{1-1}
  \cline{2-1}
  \cline{4-1}
  8 & & & \\
  \cline{1-1}
  9 & $10^9$ & & $10^{9}$\\
  \cline{1-1}
  10 & & &\\
  \hline
\end{tabular}\\
对于所有数据,保证 $2 \leq n \leq L \leq R \leq 10^9$
\begin{lstlisting}[language=C++]
#include <iostream>
#include <fstream>

using namespace std;

int main(){
	ifstream fin("candy.in");
	ofstream fout("candy.out");
	int n,l,r;
	int k;
	fin>>n>>l>>r;
	if(l<n*(r/n)&&n*(r/n)<r){
		k = n * (r/n) -1-n*(r/n-1);
	}else if(n*(r/n)<l){
		k = r - n;
	}
	fout<<k;
	return 0;
}
\end{lstlisting}

\section{\& Reference引用}
\begin{lstlisting}[language=C++]
#include <iostream>
#include <vector>

using namespace std;

void modifyVector(vector<int>& vec){
	vec.push_back(4);
}

int main(){
	vector<int> nums = {1,2,3};
	modifyVector(nums);

	for(int num : nums){
		cout<<num<<" ";
	}
	cout<<endl;
	return 0;
}
\end{lstlisting}


\chapter{文件及基本读写}
\section{fstream}
\section{ifstream}
\section{ifstream}

\chapter{std}
在C++中,"std"和"STL"都是与标准库(Standard Library)相关的概念,但它们具有不同的含义和范围。\\
1. STL(Standard Template Library):STL是C++标准库中的一部分,它是一组模板类和函数的集合,提供了丰富的数据结构和算法实现。STL包括容器(如vector、list、map等)、迭代器、算法(如排序、搜索、转换等)、函数对象等。STL的设计理念是基于泛型编程,它通过模板技术使算法和数据结构能够独立于特定类型工作,提供了高度可复用和可扩展的组件。\\
2. std(Standard Namespace):std是C++标准库中的命名空间(namespace),其中包含了大量的类、函数和常量。std命名空间用于将C++标准库中的所有标识符(如容器、算法、输入输出等)进行组织和隔离,以避免命名冲突。使用std命名空间,我们可以通过前缀"std::"访问标准库中的各种成员。\\
总结起来,STL是C++标准库的一个子集,包含了模板类和函数,提供了通用的数据结构和算法。而"std"是C++标准库的命名空间,用于组织和隔离标准库中的各个成员。STL是std命名空间的一部分,我们可以使用"std::"前缀来访问STL中的各种组件。\\
例如,使用STL中的vector容器和sort算法,我们可以这样引用:\\
\begin{lstlisting}[language=C++]
#include <vector>   // 包含STL中的vector容器
#include <algorithm>  // 包含STL中的sort算法

int main() {
    std::vector<int> nums = {4, 2, 6, 1, 3};  // 使用STL中的vector容器
    std::sort(nums.begin(), nums.end());//使用STL中的sort算法对nums进行排序

    return 0;
}
\end{lstlisting}
在上述示例中,我们使用了STL中的vector容器和sort算法,通过std命名空间访问这些组件。

\chapter{stl}

\chapter{类}

\chapter{面向对象}

\section{算术运算符}
\begin{lstlisting}[language=C++]
// 算术运算符
int arithmeticOperator(){
	int a = 5;
	int b = 3;
	int c;
	cout<<"a = "<<a<<endl;
	cout<<"b = "<<b<<endl;
	c = a + b;
	cout<<"c = a + b = "<<c<<endl;
	c = a - b;
	cout<<"c = a - b = "<<c<<endl;
	c = a * b;
	cout<<"c = a * b = "<<c<<endl;
	c = a / b;
	cout<<"c = a / b = "<<c<<endl;
	c = a % b;
	cout<<"c = a % b = "<<c<<endl;
	int d = 7;
	cout<<"d = "<<d<<endl;
	c = d++;
	cout<<"c = d++ = "<<c<<endl;
	c = d--;
	cout<<"c = d-- = "<<c<<endl;
	return 0;
}
\end{lstlisting}

\section{关系运算符}
\begin{lstlisting}[language=C++]
// 关系运算符
int relationalOperator(){
	int a = 5;
	int b = 3;
	cout<<"a = "<<a<<endl;
	cout<<"b = "<<b<<endl;
	int c;
	if (a == b){
		cout<<"a 等于 b"<<endl;
	}else{
		cout<<"a 不等于 b"<<endl;
	}
	if(a < b){
		cout<<"a 小于 b"<<endl;
	}else{
		cout<<"a 不小于 b"<<endl;
	}
	if(a > b){
		cout<<"a 大于 b"<<endl;
	}else{
		cout<<"a 不大于 b"<<endl;
	}
	return 0;
}
\end{lstlisting}

\section{逻辑运算符}
逻辑运算符在C++中用于解决以下问题:\\
1. 条件判断:逻辑运算符允许程序员在条件语句中对多个条件进行组合判断。通过使用逻辑与运算符(\&\&)和逻辑或运算符(||),可以根据多个条件的组合结果来确定程序的执行路径。\\
2. 循环控制:逻辑运算符在循环语句中起到关键作用,例如在while循环或do-while循环中,使用逻辑运算符可以设置多个条件来控制循环的执行和终止条件。\\
3. 布尔逻辑操作:逻辑运算符允许对布尔值进行操作,将多个布尔值进行组合,从而得到新的布尔值。这对于程序中的条件逻辑判断非常有用。\\
通过使用逻辑运算符,程序员可以根据条件的组合结果来进行复杂的判断和控制,从而实现程序的逻辑流程控制和条件判断。这样可以使程序更加灵活和可控,并能够处理多种不同的情况。\\
C++中的逻辑运算符用于对条件表达式进行逻辑运算,通常返回布尔值(true或false)。以下是C++中常用的逻辑运算符:\\
1. 逻辑与运算符(\&\&):当且仅当两个操作数都为true时,结果为true。否则,结果为false。\\
\begin{lstlisting}[language=C++]
   bool a = true;
   bool b = false;
   bool result = a && b; // 结果为false
\end{lstlisting}
2. 逻辑或运算符(||):当至少有一个操作数为true时,结果为true。只有当两个操作数都为false时,结果为false。
\begin{lstlisting}[language=C++]
   bool a = true;
   bool b = false;
   bool result = a || b; // 结果为true
\end{lstlisting}
3. 逻辑非运算符(!):对操作数进行取反操作,如果操作数为true,则结果为false;如果操作数为false,则结果为true。
\begin{lstlisting}[language=C++]
   bool a = true;
   bool result = !a; // 结果为false
\end{lstlisting}
逻辑运算符通常与条件语句(例如if语句和while循环)一起使用,用于控制程序的执行流程和判断条件的满足情况。\\
C++ 中的逻辑运算符包括逻辑与(\&\&)、逻辑或(||)、逻辑非(!)三种。它们的作用是对逻辑表达式进行求值,以判断表达式的真假。\\
当使用逻辑与(\&\&)时,只有当两个操作数都为真(非零)时,整个表达式才为真,否则为假。因此,如果一个操作数为真,另一个操作数为假,整个表达式的结果就是假。\\
同样的道理,当使用逻辑或(||)时,只有当两个操作数都为假(零)时,整个表达式才为假,否则为真。如果一个操作数为假,另一个操作数为真,整个表达式的结果也是真。\\
逻辑非(!)则是将操作数的真假值取反。如果操作数为真,取反后就是假;如果操作数为假,取反后就是真。\\
因此,当使用逻辑运算符时,需要注意操作数的真假值,以便正确地求出整个表达式的值。
\begin{lstlisting}[language=C++]
// 逻辑运算符
int logicalOperator(){
	int a = 3,b = 5,c;
	cout<<"a = "<<a<<endl;
	cout<<"b = "<<b<<endl;
	if (a&&b){
		cout<<"a&&b条件为 true"<<endl;
	}
	if (a || b){
		cout<<"a||b条件为 true"<<endl;
	}
	// 改变a和b的值
	a = 0;
	b = 5;
	if (a && b){
		cout<<"a&&b条件为 true"<<endl;
	}else{
		cout<<"a&&b条件为 false"<<endl;
	}
	if (!(a&&b)){
		cout<<"!(a&&b)条件为 true"<<endl;
	}
	return 0;
}
\end{lstlisting}

\chapter{注释}
\section{单行多行注释}
C++ 支持单行注释和多行注释。注释中的所有字符会被 C++ 编译器忽略。\\
// - 一般用于单行注释。\\
/* ... */ - 一般用于多行注释。

\chapter{常量}
\section{\#define预处理器}
使用 \#define 预处理器定义常量
\begin{lstlisting}[language=C++]
// #define 预处理器定义常量
#define LENGTH 3
#define WIDTH 2

int areaDefine(){
	int area;
	area = LENGTH * WIDTH;
	cout<<"area = "<<area<<endl;
	return 0;
}
\end{lstlisting}

\section{const 关键字}
\begin{lstlisting}[language=C++]
// 使用 const 前缀声明指定类型的常量
int constConstant(){
	const int LENGTH_ = 3;
	const int WIDTH_ = 2;
	int area;
	area = LENGTH_ * WIDTH_;
	cout<<"area = "<<area<<endl; 
	return 0;
}
\end{lstlisting}

\chapter{修饰符类型}
\begin{lstlisting}[language=C++]
int modifier(){
	short int i; // 有符号短整数
	short unsigned int j;
	j = 50000;
	i = j;
	cout<<"j = "<<j<<endl;	
	cout<<"i = "<<i<<endl;	
	return 0;
}
\end{lstlisting}

\chapter{函数}
\section{CSP-J(普及组)2022年T1乘方(pow)}

\chapter{数组}

\chapter{STL}
\section{vector向量}
\subsection{stl\_vector.h}
\begin{lstlisting}[language=C++]
C:\Program Files\CodeBlocks\MinGW\lib\gcc\x86_64-w64-mingw32\8.1.0
\include\c++\bits\stl_vector.h
\end{lstlisting}
\begin{lstlisting}[language=C++]
#include <iostream>
#include <vector>

using namespace std;

int main() {
    // 创建一个整数向量
    vector<int> numbers;

    // 向向量中添加元素
    numbers.push_back(1);
    numbers.push_back(2);
    numbers.push_back(3);

    // 访问和修改向量中的元素
    cout << "First element: " << numbers[0] << endl;
    cout << "Second element: " << numbers[1] << endl;

    numbers[2] = 4;

    // 遍历向量中的元素
    cout << "All elements: ";
    for (int i = 0; i < numbers.size(); ++i) {
        cout << numbers[i] << " ";
    }
    cout << endl;

    // 使用迭代器遍历向量中的元素
    cout << "All elements (using iterator): ";
    for (vector<int>::iterator it = numbers.begin(); it != numbers.end(); ++it) {
        cout << *it << " ";
    }
    cout << endl;

    // 删除向量中的最后一个元素
    numbers.pop_back();

    // 检查向量是否为空
    if (numbers.empty()) {
        cout << "Vector is empty" << endl;
    } else {
        cout << "Vector is not empty" << endl;
    }

    // 清空向量
    numbers.clear();

    // 检查向量是否为空
    if (numbers.empty()) {
        cout << "Vector is empty" << endl;
    } else {
        cout << "Vector is not empty" << endl;
    }

    return 0;
}
\end{lstlisting}

\chapter{class类}
在C++中,`class`关键字用于定义一个类。类是一种用户自定义的数据类型,用于封装数据和操作。类可以包含成员变量(属性)和成员函数(方法),
用于描述对象的状态和行为。

以下是一个简单的C++类的示例:

\begin{lstlisting}[language=C++]
#include <iostream>
using namespace std;

class Rectangle {
private:
    int length;
    int width;

public:
    // 构造函数
    Rectangle(int l, int w) : length(l), width(w) {}

    // 成员函数
    int getArea() {
        return length * width;
    }

    // 成员函数
    int getPerimeter() {
        return 2 * (length + width);
    }
};

int main() {
    // 创建一个 Rectangle 对象
    Rectangle rect(5, 3);

    // 调用对象的成员函数
    int area = rect.getArea();
    int perimeter = rect.getPerimeter();

    // 打印结果
    cout << "Area: " << area << endl;
    cout << "Perimeter: " << perimeter << endl;

    return 0;
}
\end{lstlisting}
在上述示例中,定义了一个名为`Rectangle`的类,它具有私有成员变量`length`和`width`,以及公有成员函数`getArea()`和`getPerimeter()`。构造函数用于初始化对象的数据成员。
在`main()`函数中,创建了一个`Rectangle`对象`rect`,并使用对象的成员函数`getArea()`和`getPerimeter()`计算矩形的面积和周长。最后,使用`std::cout`打印结果。
类提供了一种组织和封装相关数据和行为的方式,使代码更加模块化和可维护。通过类的实例化,可以创建多个对象,每个对象都有自己的数据和方法。
除了成员变量和成员函数,类还可以包含访问修饰符(如`public`、`private`、`protected`)和其他特性(如继承、多态)等。

\section{构造函数}
在C++中,构造函数是一种特殊的成员函数,用于在创建对象时进行初始化操作。构造函数的名称与类的名称相同,并且没有返回类型(包括`void`)。它可以有参数,
也可以没有参数。\\
构造函数在以下几种情况下会被自动调用:\\
1. 在创建对象时,使用`new`运算符动态分配内存时。\\
2. 在声明对象时,使用类的默认构造函数进行初始化。\\
3. 在将一个对象作为参数传递给函数时,调用拷贝构造函数进行复制。\\
以下是一个简单的示例,展示了如何定义和使用构造函数:
\begin{lstlisting}[language=C++]
#include <iostream>

class MyClass {
private:
    int value;

public:
    // 默认构造函数
    MyClass() {
        value = 0;
        std::cout << "Default constructor called" << std::endl;
    }

    // 带参数的构造函数
    MyClass(int val) {
        value = val;
        std::cout << "Parameterized constructor called" << std::endl;
    }

    // 成员函数
    int getValue() {
        return value;
    }
};

int main() {
    // 使用默认构造函数创建对象
    MyClass obj1;
    std::cout << "Value: " << obj1.getValue() << std::endl;

    // 使用带参数的构造函数创建对象
    MyClass obj2(10);
    std::cout << "Value: " << obj2.getValue() << std::endl;

    return 0;
}
\end{lstlisting}
在上述示例中,定义了一个名为`MyClass`的类,其中包含一个私有成员变量`value`和三个构造函数。默认构造函数用于初始化`value`为0,带参数的构造函数用于将传入的值赋给`value`。\\
在`main()`函数中,首先使用默认构造函数创建了一个`MyClass`对象`obj1`,并通过`getValue()`方法获取对象的值并打印。然后,使用带参数的构造函数创建了另一个对象`obj2`,同样获取并打印了对象的值。\\
构造函数在对象创建时自动调用,用于进行必要的初始化工作。你可以根据需要定义不同的构造函数,以支持不同的初始化方式。\\
希望这个示例对你有帮助!如果你还有其他问题,请随时提问。

\section{拷贝构造函数}
在C++中,拷贝构造函数(Copy Constructor)是一种特殊的构造函数,用于创建一个对象的副本。拷贝构造函数通常以传入对象的引用作为参数,
并使用该对象的数据来初始化新对象。\\
拷贝构造函数在以下情况下会被自动调用:\\
1. 在将一个对象作为参数传递给函数时,进行参数的复制。\\
2. 在使用一个对象初始化另一个对象时,进行对象的复制。\\
3. 在函数返回一个对象时,进行对象的复制。\\
以下是一个简单的示例,展示了如何定义和使用拷贝构造函数:
\begin{lstlisting}[language=C++]
#include <iostream>

class MyClass {
private:
    int value;

public:
    // 默认构造函数
    MyClass() {
        value = 0;
        std::cout << "Default constructor called" << std::endl;
    }

    // 带参数的构造函数
    MyClass(int val) {
        value = val;
        std::cout << "Parameterized constructor called" << std::endl;
    }

    // 拷贝构造函数
    MyClass(const MyClass& other) {
        value = other.value;
        std::cout << "Copy constructor called" << std::endl;
    }

    // 成员函数
    int getValue() {
        return value;
    }
};

void printObject(const MyClass& obj) {
    std::cout << "Object value: " << obj.getValue() << std::endl;
}

int main() {
    // 使用默认构造函数创建对象
    MyClass obj1;
    std::cout << "Value: " << obj1.getValue() << std::endl;

    // 使用带参数的构造函数创建对象
    MyClass obj2(10);
    std::cout << "Value: " << obj2.getValue() << std::endl;

    // 使用拷贝构造函数创建对象的副本
    MyClass obj3 = obj2;
    std::cout << "Value: " << obj3.getValue() << std::endl;

    // 作为函数参数传递对象
    printObject(obj3);

    return 0;
}
\end{lstlisting}
在上述示例中,`MyClass`类定义了默认构造函数、带参数的构造函数和拷贝构造函数。拷贝构造函数以传入对象的引用作为参数,并将传入对象的值复制给新对象的成员变量。\\
在`main()`函数中,首先使用默认构造函数创建了一个`MyClass`对象`obj1`,然后使用带参数的构造函数创建了另一个对象`obj2`。接下来,使用拷贝构造函数将`obj2`复制到新对象`obj3`。最后,通过调用`printObject()`函数将`obj3`作为参数传递给函数。\\
拷贝构造函数在对象的复制过程中起到重要作用,确保新对象与原始对象具有相同的值。如果没有显式定义拷贝构造函数,编译器会自动生成一个默认的拷贝构造函数。\\
需要注意的是,拷贝构造函数的参数通常是`const`引用,以防止在拷

\clearpage
\end{document}
