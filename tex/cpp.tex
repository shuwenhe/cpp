%%%%%%%%%%%%%%%%%%%%%%%%%%%%%%%%%%%%%%%%%%%%%%%%%%%%%%%%%%%%%%%%%%%%%%%%%%%%%%%%%%%%%%%%%%%%%%%%%%%%%%%%%%
%Write by:ShuwenHe
%Date:20230613
%%%%%%%%%%%%%%%%%%%%%%%%%%%%%%%%%%%%%%%%%%%%%%%%%%%%%%%%%%%%%%%%%%%%%%%%%%%%%%%%%%%%%%%%%%%%%%%%%%%%%%%%%%

%%%%%%%%%%%%%%%%%%%%%%%%%%%%%%%%%%%%%%%%%%%%%%%%%%%%%%%%%%%%%%%%%%%%%%%%%%%%%%%%%%%%%%%%%%%%%%%%%%%%%%%%%%
\documentclass[12pt,twiside,a4paper]{ctexbook}
\usepackage[centertags]{amsmath}
\usepackage{amsfonts}
\usepackage{amsthm}
\usepackage{newlfont}
\usepackage{makeidx}
\usepackage{wasysym}
\usepackage{geometry} 
\usepackage{graphics}
\usepackage{slashbox} 
\usepackage{fancyhdr} 
\usepackage[pdftex]{graphicx}
\usepackage{epstopdf}
\usepackage{cite}
\usepackage{listings}

\usepackage[numbers,sort&compress]{natbib}

\setlength\parskip{\baselineskip}
\setcounter{tocdepth}{0}
\setcounter{secnumdepth}{4}
\renewcommand\thesection{\arabic{section}}
\usepackage[pdfstartview=FitH,CJKbookmarks=true,bookmarks,bookmarksnumbered=true,
    colorlinks=true,citecolor=black,linkcolor=black,anchorcolor=green,urlcolor=black]{hyperref}
\usepackage{titlesec}
\titleformat{\chapter}[display]{\normalfont\huge\bfseries\center}{\chaptertitlename}{1pt}{\Huge}
\titleformat{\section}{\normalfont\Large\bfseries}{\thesection}{1em}{}
\titleformat{\subsection}{\normalfont\large\bfseries}{\thesubsection}{1em}{}
\titleformat{\subsubsection}{\normalfont\normalsize\bfseries}{\thesubsubsection}{1em}{}
\titleformat{\paragraph}[runin]{\normalfont\normalsize\bfseries}{\theparagraph}{1em}{}
\titleformat{\subparagraph}[runin]{\normalfont\normalsize\bfseries}{\thesubparagraph}{1em}{}
\titlespacing*{\chapter} {0pt}{10pt}{10pt}
\titlespacing*{\section} {0pt}{0.5ex plus 1ex minus .2ex}{0.3ex plus .2ex}
\titlespacing*{\subsection} {0pt}{0.25ex plus 1ex minus .1ex}{0.5ex plus .1ex}
\titlespacing*{\subsubsection}{0pt}{3.25ex plus 1ex minus .2ex}{1.5ex plus .2ex}
\titlespacing*{\paragraph} {0pt}{3.25ex plus 1ex minus .2ex}{1em}
\titlespacing*{\subparagraph} {\parindent}{3.25ex plus 1ex minus .2ex}{1em}
\numberwithin{chapter}{part}
\geometry{left=2.0cm,right=20mm,top=25mm,bottom=25mm}
\let\cleardoublepage\clearpage
%%%%%%%%%%%%%%%%%%%%%%%%%%%%%%%%%%%%%%%%%%%%%%%%%%%%%%%%%%%%%%%%%%%%%%%%%%%%%%%%%%%%%%%%%%%%%%%%%%%%%%%%%%

%%%%%%%%%%%%%%%%%%%%%%%%%%%%%%%%%%%%%%%%%%%%%%%%%%%%%%%%%%%%%%%%%%%%%%%%%%%%%%%%%%%%%%%%%%%%%%%%%%%%%%%%%%
%mathematics
\usepackage{amssymb}
\usepackage{diagbox}
%%%%%%%%%%%%%%%%%%%%%%%%%%%%%%%%%%%%%%%%%%%%%%%%%%%%%%%%%%%%%%%%%%%%%%%%%%%%%%%%%%%%%%%%%%%%%%%%%%%%%%%%%%

%%%%%%%%%%%%%%%%%%%%%%%%%%%%%%%%%%%%%%%%%%%%%%%%%%%%%%%%%%%%%%%%%%%%%%%%%%%%%%%%%%%%%%%%%%%%%%%%%%%%%%%%%%
%
%%%%%%%%%%%%%%%%%%%%%%%%%%%%%%%%%%%%%%%%%%%%%%%%%%%%%%%%%%%%%%%%%%%%%%%%%%%%%%%%%%%%%%%%%%%%%%%%%%%%%%%%%%

%%%%%%%%%%%%%%%%%%%%%%%%%%%%%%%%%%%%%%%%%%%%%%%%%%%%%%%%%%%%%%%%%%%%%%%%%%%%%%%%%%%%%%%%%%%%%%%%%%%%%%%%%%
%
\usepackage{tipa}
%%%%%%%%%%%%%%%%%%%%%%%%%%%%%%%%%%%%%%%%%%%%%%%%%%%%%%%%%%%%%%%%%%%%%%%%%%%%%%%%%%%%%%%%%%%%%%%%%%%%%%%%%%

%%%%%%%%%%%%%%%%%%%%%%%%%%%%%%%%%%%%%%%%%%%%%%%%%%%%%%%%%%%%%%%%%%%%%%%%%%%%%%%%%%%%%%%%%%%%%%%%%%%%%%%%%%
\begin{document}
%%%%%%%%%%%%%%%%%%%%%%%%%%%%%%%%%%%%%%%%%%%%%%%%%%%%%%%%%%%%%%%%%%%%%%%%%%%%%%%%%%%%%%%%%%%%%%%%%%%%%%%%%%

\author
{
Peking University\\
北京大学\\
ShuwenHe\\
何书文
}

%%%%%%%%%%%%%%%%%%%%%%%%%%%%%%%%%%%%%%%%%%%%%%%%%%%%%%%%%%%%%%%%%%%%%%%%%%%%%%%%%%%%%%%%%%%%%%%%%%%%%%%%%%
\title{C++基础}
\maketitle
\tableofcontents
\pagestyle{fancy}
%%%%%%%%%%%%%%%%%%%%%%%%%%%%%%%%%%%%%%%%%%%%%%%%%%%%%%%%%%%%%%%%%%%%%%%%%%%%%%%%%%%%%%%%%%%%%%%%%%%%%%%%%%

\lhead{\includegraphics{shuwenedu.png}}
\rhead{科技特长生升学规划}
\lfoot{\includegraphics{pku.png}算法第一人北大何书文}
\rfoot{改变您家孩子命运的老师}
%%%%%%%%%%%%%%%%%%%%%%%%%%%%%%%%%%%%%%%%%%%%%%%%%%%%%%%%%%%%%%%%%%%%%%%%%%%%%%%%%%%%%%%%%%%%%%%%%%%%%%%%%%

%%%%%%%%%%%%%%%%%%%%%%%%%%%%%%%%%%%%%%%%%%%%%%%%%%%%%%%%%%%%%%%%%%%%%%%%%%%%%%%%%%%%%%%%%%%%%%%%%%%%%%%%%%
\chapter{环境搭建}
\section{linux命令}
\subsection{linux文件文件夹命令}
创建文件夹\\
mkdir cpp创建文件夹\\
mkdir cpp\\
进入文件夹\\
cd cpp\\
vi richard.cpp\\
i进入insert模式\\
o进入下一行\\
int 整型\\
main()主函数function\\
// 单行注释\\
“”输入字符串\\
endl\\
;一行语句结束需要用分号结束\\
return 0 一个函数正确执行完成之后需要用return 0来结束\\
esc推出vi编辑器\\
:wq保存并推出 write quit\\
ls查看当前文件夹有什么文件ls - list directory contents

\section{文本编辑器}

\section{C++编译器}
命令行使用下面的命令来检查您的系统上是否安装了gcc\\
g++ -v
g++ 编译器C++\\
使用 -o 选项指定可执行程序的文件名\\
g++ hello.cpp -o hello\\
./richard 执行编译器编译产生的二进制文件\\
指定使用C++14来编译\\
g++ -std=c++14 cpp.cpp -o cpp 

\subsection{g++常用命令选项}
-o\\
file生成指定的输出文件,用在生成可执行文件时。\\
C++ 中的分号\&语句块\\
编辑编译执行C++程序

\chapter{C++基本语法}
\section{}
\begin{lstlisting}[language=C++]
#include <iostream>
using namespace std;
int main() {
	cout<<"hello"<<endl;
	return 0;
}
\end{lstlisting}
C++头文件<iostream>\\
using namespace std; 告诉编译器使用std命名空间。\\
int main()是主函数,程序从这里开始执行。\\
cout<<"Hello World"; 会在屏幕上显示消息 "Hello World"。\\
下一行 return 0; 终止 main( )函数,并向调用进程返回值 0。\\
在 C++ 中,分号是语句结束符。也就是说,每个语句必须以分号结束。它表明一个逻辑实体的结束。

\section{C++ 标识符}

\section{C++ 关键字}

\chapter{C++ 注释}
\section{}
C++ 支持单行注释和多行注释。注释中的所有字符会被 C++ 编译器忽略。\\
// - 一般用于单行注释。\\
/* ... */ - 一般用于多行注释。

\chapter{C++ 数据类型}
使用编程语言进行编程时,需要用到各种变量来存储各种信息。变量保留的是它所存储的值的内存位置。这意味着,当您创建一个变量时,就会在内存中保留一些空间。\\
您可能需要存储各种数据类型(比如字符型、宽字符型、整型、浮点型、双浮点型、布尔型等)的信息,操作系统会根据变量的数据类型,来分配内存和决定在保留内存中存储什么。
\section{基本的内置类型}
\begin{table}[h]
    \centering
    \caption{几种基本的 C++ 数据类型}
    \label{tab:example}
    \begin{tabular}{|c|c|}
    \hline
    \textbf{类型} & \textbf{关键字}\\
    \hline
    布尔型 & bool\\
    \hline
    字符型 & char\\
    \hline
    整型 & int\\
    \hline
    浮点型 & float\\
    \hline
    双浮点型 & double\\
    \hline
    无类型 & void\\
    \hline
    宽字符型 & wchar\_t\\
    \hline
    \end{tabular}
\end{table}
一些基本类型可以使用一个或多个类型修饰符进行修饰:\\
signed\\
unsigned\\
short\\
long\\
下面实例会输出电脑上各种数据类型的大小。\\
\begin{lstlisting}[language=C++]
    #include<iostream>  
    #include <limits>
     
    using namespace std;  
      
    int main()  
    {  
        cout << "type: \t\t" << "************size**************"<< endl;  
        cout << "bool: \t\t" << "所占字节数:" << sizeof(bool);  
        cout << "\t最大值:" << (numeric_limits<bool>::max)();  
        cout << "\t\t最小值:" << (numeric_limits<bool>::min)() << endl;  
        cout << "char: \t\t" << "所占字节数:" << sizeof(char);  
        cout << "\t最大值:" << (numeric_limits<char>::max)();  
        cout << "\t\t最小值:" << (numeric_limits<char>::min)() << endl;  
        cout << "signed char: \t" << "所占字节数:" << sizeof(signed char);  
        cout << "\t最大值:" << (numeric_limits<signed char>::max)();  
        cout << "\t\t最小值:" << (numeric_limits<signed char>::min)() << endl;  
        cout << "unsigned char: \t" << "所占字节数:" << sizeof(unsigned char);  
        cout << "\t最大值:" << (numeric_limits<unsigned char>::max)();  
        cout << "\t\t最小值:" << (numeric_limits<unsigned char>::min)() << endl;  
        cout << "wchar_t: \t" << "所占字节数:" << sizeof(wchar_t);  
        cout << "\t最大值:" << (numeric_limits<wchar_t>::max)();  
        cout << "\t\t最小值:" << (numeric_limits<wchar_t>::min)() << endl;  
        cout << "short: \t\t" << "所占字节数:" << sizeof(short);  
        cout << "\t最大值:" << (numeric_limits<short>::max)();  
        cout << "\t\t最小值:" << (numeric_limits<short>::min)() << endl;  
        cout << "int: \t\t" << "所占字节数:" << sizeof(int);  
        cout << "\t最大值:" << (numeric_limits<int>::max)();  
        cout << "\t最小值:" << (numeric_limits<int>::min)() << endl;  
        cout << "unsigned: \t" << "所占字节数:" << sizeof(unsigned);  
        cout << "\t最大值:" << (numeric_limits<unsigned>::max)();  
        cout << "\t最小值:" << (numeric_limits<unsigned>::min)() << endl;  
        cout << "long: \t\t" << "所占字节数:" << sizeof(long);  
        cout << "\t最大值:" << (numeric_limits<long>::max)();  
        cout << "\t最小值:" << (numeric_limits<long>::min)() << endl;  
        cout << "unsigned long: \t" << "所占字节数:" << sizeof(unsigned long);  
        cout << "\t最大值:" << (numeric_limits<unsigned long>::max)();  
        cout << "\t最小值:" << (numeric_limits<unsigned long>::min)() << endl;  
        cout << "double: \t" << "所占字节数:" << sizeof(double);  
        cout << "\t最大值:" << (numeric_limits<double>::max)();  
        cout << "\t最小值:" << (numeric_limits<double>::min)() << endl;  
        cout << "long double: \t" << "所占字节数:" << sizeof(long double);  
        cout << "\t最大值:" << (numeric_limits<long double>::max)();  
        cout << "\t最小值:" << (numeric_limits<long double>::min)() << endl;  
        cout << "float: \t\t" << "所占字节数:" << sizeof(float);  
        cout << "\t最大值:" << (numeric_limits<float>::max)();  
        cout << "\t最小值:" << (numeric_limits<float>::min)() << endl;  
        cout << "size_t: \t" << "所占字节数:" << sizeof(size_t);  
        cout << "\t最大值:" << (numeric_limits<size_t>::max)();  
        cout << "\t最小值:" << (numeric_limits<size_t>::min)() << endl;  
        cout << "string: \t" << "所占字节数:" << sizeof(string) << endl;  
        // << "\t最大值:" << (numeric_limits<string>::max)() << "\t最小值:" << (numeric_limits<string>::min)() << endl;  
        cout << "type: \t\t" << "************size**************"<< endl;  
        return 0;  
    }
\end{lstlisting}


\chapter{控制语句}
\section{for循环}
// 高斯求和公式求和1+2+3+...+100
\begin{lstlisting}[language=C++]
#include <iostream>
using namespace std;
int main() {
    int sum = 0;
    sum = (1+100)*100/2;
    cout << "gauss sum=" << sum << endl;
    return 0;
}
\end{lstlisting}

\section{}
for循环求和1+2+3+...+100\\
\begin{lstlisting}[language=C++]
int forSum(){
    int sum = 0;
    for (int i = 1; i <= 100; i++) {
        sum += i;
    }
    cout << "for sum=" << sum << endl;
    return 0;
}
\end{lstlisting}

\section{2014NOIP普及T1珠心算测验}
【题目描述】\\
珠心算是一种通过在脑中模拟算盘变化来完成快速运算的一种计算技术。珠心算训练,既能够开发智力,又能够为日常生活带来很多便利,因而在很多学校得到普及。
某学校的珠心算老师采用一种快速考察珠心算加法能力的测验方法。他随机生成一个正整数集合,集合中的数各不相同,然后要求学生回答:其中有多少个数,恰好等于集合中另外两个(不同的)数之和?
最近老师出了一些测验题,请你帮忙求出答案。\\
输入格式\\
共两行,第一行包含一个整数
n,表示测试题中给出的正整数个数。
第二行有
n个正整数,每两个正整数之间用一个空格隔开,表示测试题中给出的正整数。
输出格式
一个整数,表示测验题答案。
【输入输出样例】\\
输入\\
复制\\
4\\
1 2 3 4\\
输出1\\
复制\\
2\\
【样例解释】\\
由\\
1+2=3,1+3=4,故满足测试要求的答案为2。\\
注意,加数和被加数必须是集合中的两个不同的数。\\
数据范围\\
3≤n≤100,测验题给出的正整数大小不超过\\
假设集合中有 $n$ 个数,我们可以使用双重循环枚举集合中的每一对数,判断它们的和是否也在集合中出现过。这个方法的时间复杂度为 $O(n^2)$,对于较大的 $n$ 可能会超时。
更高效的方法是,先将集合中的数按从小到大排序,然后对于每个数 $x$,使用双指针法在剩余的数中寻找两个数,使它们的和等于 $x$。具体地,我们可以将左指针指向 $x$ 的下一个数,将右指针指向集合中最大的数,然后不断地将左指针右移或右指针左移,直到两个指针相遇为止。如果两个指针指向的数的和等于 $x$,则找到了一组符合条件的数对。
这个方法的时间复杂度为 $O(n \log n)$(排序的时间复杂度为 $O(n \log n)$,每个数最多被判断一次,因此双指针法的时间复杂度为 $O(n)$),可以通过本题。
以下是使用双指针法实现题目的 C++ 代码:
\begin{lstlisting}[language=C++]
#include <iostream>
#include <vector>
#include <algorithm>

using namespace std;

int main() {
    int n;
    cin >> n;
    vector<int> a(n);
    for (int i = 0; i < n; i++) {
        cin >> a[i];
    }
    sort(a.begin(), a.end());
    int cnt = 0;
    for (int i = 0; i < n; i++) {
        int left = i + 1, right = n - 1;
        while (left < right) {
            int sum = a[left] + a[right];
            if (sum == a[i]) {
                cnt++;
                left++;
                right--;
            } else if (sum < a[i]) {
                left++;
            } else {
                right--;
            }
        }
    }
    cout << cnt << endl;
    return 0;
}
\end{lstlisting}
代码中使用了 STL 的 `vector` 和 `sort`,可以方便地实现数组的排序。双指针法的部分使用了 `while` 循环和条件语句进行实现。

\chapter{函数}
\section{CSP-J(普及组)2022年T1乘方(pow)}

\chapter{数组}

\chapter{STL}

\clearpage
\end{document}
